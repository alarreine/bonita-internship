\documentclass[12pt]{article}
\usepackage[utf8x]{inputenc}
\usepackage{hyperref}
\usepackage{amsmath}
\usepackage{graphicx}
\usepackage[colorinlistoftodos]{todonotes}
\usepackage[toc,page]{appendix}
\usepackage{smartdiagram}
\usepackage{indentfirst}

\usepackage[edges]{forest}
\usetikzlibrary{fit}


\begin{document}
\graphicspath{{images/}}
\begin{titlepage}

\newcommand{\HRule}{\rule{\linewidth}{0.5mm}} % Defines a new command for the horizontal lines, change thickness here
\setlength{\topmargin}{0in}
\center % Center everything on the page


 \begin{minipage}{0.4\textwidth}
\begin{flushleft} \large
\hspace*{-0.5cm}
\includegraphics[scale=0.4]{images/imag.jpg}\\
\end{flushleft}
\end{minipage}
~
\begin{minipage}{0.5\textwidth}
\begin{flushright} \large
\hspace*{2cm}
\includegraphics[scale=0.4]{images/company.png}\\
\end{flushright}
\end{minipage}\\[1cm]
%----------------------------------------------------------------------------------------
%	HEADING SECTIONS
%----------------------------------------------------------------------------------------

\textsc{\LARGE Université Grenoble Alpes}\\[1.5cm]
\textsc{\Large Master 2 Génie Informatique}\\[0.5cm]
\textsc{\large Mémoire d'Alternance}\\[0.5cm]

%----------------------------------------------------------------------------------------
%	TITLE SECTION
%----------------------------------------------------------------------------------------

\HRule \\[0.4cm]
{ \huge \bfseries Ingénieur DevOps}\\[0.4cm]
\HRule \\[1cm]

%----------------------------------------------------------------------------------------
%	AUTHOR SECTION
%----------------------------------------------------------------------------------------

\begin{minipage}{0.4\textwidth}
\begin{flushleft} \large
\emph{Author:}\\
Gerardo \textsc{LARREINEGABE} \\
\end{flushleft}
\end{minipage}
~
\begin{minipage}{0.5\textwidth}
\begin{flushright} \large
\emph{Maître d’apprentissage:} \\
Duy \textsc{Tran Quang} \\[0.5cm] % Supervisor's Name
\emph{Tuteur Pédagogique:} \\
Olivier \textsc{Gruber} % Supervisor's Name
\end{flushright}
\end{minipage}\\[1cm]

%----------------------------------------------------------------------------------------
%	DATE SECTION
%----------------------------------------------------------------------------------------

{\large Grenoble, le 25 août 2018}\\[0.5cm]


\vfill % Fill the rest of the page with whitespace

\end{titlepage}

\renewcommand{\abstractname}{Résumé}
\begin{abstract}
\noindent
Dans le cadre de mon alternance, j'ai intégré l'équipe de IT au sein de
la société Bonitasoft, qui gère plusieurs projets dont BCD forme part et vise
à fournir au client une solution DevOps et des bonnes pratiques pour aboutir
à la Livraison Continue des applications créés avec Bonitasoft.
Le nom BCD est l'acronyme en anglais de \textit{Bonita Continuous Delivery}. \\
BCD aborde deux problème commun des développeur qui est la \textbf{création}
d'un environnement normalisé et le \textbf{déploiement} d'une application
Bonitasoft.\\ \\
\textbf{\textit{Mot clés:} livraison continue, déploiement,
création d’environnement, devops}
\end{abstract}

\newpage
\tableofcontents
\newpage

\section{Introduction}
Dans le cadre de mon alternance, j'ai occupé un poste d'Ingénieur DevOps dans
l'entreprise Bonitasoft.
Pour commencer, je présenterai mon entreprise, l'organisation, les différentes équipes,
la méthode de travail de façon générale.
En deuxième partie, je mettrai un accent sur l'équipe avec laquelle j'ai collaboré,
l'organisation interne, la façon de travailler et les activités que l'équipe
mène tous les jours.
En troisième, j'exposerai le projet sur lequel j'ai travaillé, la problématique,
et les objectifs de mon sujet, ainsi que les différentes techniques requises pour
les accomplir.

\section{Environnement Professionnel}
\subsection{L'entreprise}
\subsubsection{Mission-Vision}
De la démocratisation du BPM à la transformation numérique en profondeur de l'entreprise, nous sommes à la pointe d'une révolution professionnelle depuis que notre entreprise a vu le jour.
Le succès de nos clients est notre succès, et tout le monde chez Bonitasoft est un contributeur, chaque jour.
\cite{Bonitasoft2017BONITA}

\subsection{Un peu d'Histoire}
Bonitasoft est un éditeur de logiciels Open Source créé en 2001. Il a commencé comme Bonita BPM dans l'Institut National de Recherche en Informatique et en Automatique (INRIA) puis a été transféré au Groupe Bull.
En 2009, Miguel Valdés Faura rejoint Charles Souillard et Rodrigue Le Gall pour fonder Bonitasoft.
En 2011, après une série de levées de fonds, Bonitasoft a pu commercialiser son produit Bonita à l'international et le premier bureau de Bonitasoft a ouvert aux Etats-Unis.\cite{wikipedia_2018}

\subsection{Organisation}
Bonitasoft a quatre bureaux principaux, aux États-Unis, à San Francisco et à New York, en France à Paris et à Grenoble.
Les activités sont organisées et divisées en 5 differents services:

\begin{center}
  \smartdiagramset{bubble node size=4cm, bubble center node size=6cm}
  \smartdiagram[bubble diagram]{Bonitasoft,
  IT,Service Professionnel,
  R\&D, Support, Succès du Client}
\end{center}

Nous pouvons regarder l'organigramme dans l'Appendice \ref{appen:organigramme_general}.

\subsubsection{IT}
Le département de l'IT est le gardian principal de tout le tâches IT commun comme:
\begin{itemize}
  \item La gestion des environnements virtuels. (cloud ou serveur physique).
  \item La gestion de l'automatisation de processus de CI interne.
  \item La gestion des licences des applications payent (OS, applications).
  \item La gestion des appareils, périphériques, les ordinateurs portables,
  \item L'architectures, méthodologies et règles régissant l'utilisation et le stockage des données.
  \item L'administration de systèmes: la configuration, gestion, entretien et dépanne des environnement
  informatique multi-utilisateur (cloud ou physique).
  \item  La gestion ou l'aide aux autres département avec la culture et pratiques DevOps.
\end{itemize}

Aussi, l'équipe d'IT de Bonitasoft gère d'autres projets comme:
\begin{itemize}
  \item Salesforce: depuis la création des use cases jusqu'à la mise en production en garantissant la gouvernance du système avec une approche DevOps.
  \item BCD: c'est une solution fournit pour utiliser les bonnes pratiques DevOps pour la livraison continue (CI) d'une application Bonita.
\end{itemize}

L'équipe est composée de quatre personnes, dont l'une est moi.

\begin{figure}[h]
  \includegraphics[width=\linewidth]{it_team.png}
   \caption{Organigramme.}
   \label{figure:organigrame}
\end{figure}

\subsubsection{R\&D}
C'est le plus grand département et son travail principal est le développement de la plate-forme Bonitasoft.
Il est divisé par équipes de travail ou projets:
\begin{itemize}
  \item Back-end
  \item Front-end
  \item BICI, c'est une application autonome connectée au moteur de Bonita qui permet d'analyser l'exécution
   des processus et prévoir et améliorer l'efficacité de l'équipe.
\end{itemize}

\subsubsection{Customer Success (CS)}
Le département gère la relation entre Bonitasoft et ses clients. L'objectif de la réussite du client est de rendre le client le plus performant possible.
Cette fonction est la plus couramment utilisée dans le monde du logiciel et la plus répandue parmi les sociétés de services. Parce que le succès du client est un domaine naissant, son alignement organisationnel et ses activités sont toujours en évolution.

\subsubsection{Support}
Ce département est en charge de traiter et de diriger les différentes questions
et problématiques du client.
Chaque problème ou bouge est chargé dans l'outil Jira et traité par le département
R\&D ou l'IT.
Chaque ticket chargé a une priorité et il est résolu en ordre de priorité.

\subsubsection{Service Professionnel}
C'est une équipe d'experts qui offre de la valeur pour le succès du client en
suivant les projets depuis le début avec la meilleure expérience utilisateur possible.

\subsection{Tous ensemble}

Toutes les équipes dans l'entreprise travaillent dans un rythme Agile, et chaque équipe choisi l'implémentation qui le convient le plus.

D'un façon global, Bonitasoft travail avec le framework de la figure \ref{frame_safe}

\begin{figure}[!ht]
\centering
\includegraphics[width=\textwidth,keepaspectratio]{safe_framework.png}
\caption{Cadre de Travail \cite{safe}}
\label{frame_safe}
\end{figure}

Nous voyons dans la partie supérieur que la vision et les fonctionnalités et l'architecture de haut niveau sont définis dans le comité produit, le CEO.
Au milieu, le portfolio est traité et organisé par le Product Manager et à la fin il est développer par le team avec un Product Owner et le Scrum Mater qui dirigent et QA qui assure la qualité du code.



\section{Products Overview}
\subsection{BonitaSoft}
\begin{quotation}
BPM is the discipline of managing processes(rather than tasks) as the means for improving business performance outcomes and operational agility. \cite{gartnerdic}
\end{quotation}

Bonitasoft offer an BPM development environment to create \nameref{businessApp} based in MVC paradigm

\paragraph{Business Applications} \label{businessApp}
Any software or set of computer programs that are used by business users to perform various business functions. These business applications are used to increase productivity, to measure productivity and to perform business functions accurately.\cite{BusinessSoftware}

These applications are:
\begin{itemize}
\item Process-centric
\item Interact with internal and external systems
\item Dispatch and optimize the work between different users
\item Track and monitor events
\item Reduce application maintenance time and cost
\item Provide user interfaces where end-users/administrators interact with processes to get the job done
\end{itemize}

\subsubsection{Bonita Platform architecture}
\begin{figure}[!ht]
\includegraphics[scale=0.5]{bmp_archi.png}
\caption{Bonita Architecture}
\end{figure}

\subsubsection{Main components}
Bonita fallows the MVC paradigm, where each component is developed separately

\begin{figure}[!ht]
\includegraphics[scale=0.5]{mvc_diagram.png}
\caption{Model View Controller}
\end{figure}


\subsubsection{Composants principaux}



\subsection{Plate-forme Bonita}
ESCRIBIR
\subsection{Bonita Continuous Delivery (BCD)} \label{bcd}
Bonita Continuos Delivery ou \textit{BCD} est une solution qui permet l'intégration continue d'une application Bonita en promouvant aussi des bonnes pratiques \emph{DevOps}

Au début, BCD a démarré à partir d'un besoin interne de tester plus facilement des développements. La mise en place d'un environnement avec l'installation de \textit{Bonita Engine} était un processus lent et répétitif, aussi, suite à divers retours des clients, la direction a décidé de voir comme un potentiel produit.

Actuellement, \textit{BCD} a deux parties. Voir Figure \ref{fig:bcd_cap}:
\begin{itemize}
  \item Livingapp, qui permet la compilation d'un projet avec la gestion de bibliothèques essentielles pour une application \textit{Bonita Subscription}, sans avoir besoin de Bonita Studio, et aussi de déployer l'application compilé dans un stack Bonita configuré.
  \item Stack, qui permet d'un part la création des environnements virtuels dans le cloud (AWS actuellement), et l’installation d'un Stack Bonita dans les serveurs configurés.
\end{itemize}

\begin{figure}[!ht]
\centering
\includegraphics[width=\textwidth,keepaspectratio]{bcd_capabilities.png}
\caption{Fonctionnalités BCD}
\label{fig:bcd_cap}
\end{figure}

\subsection{Intelligent Continuous Improvement (ICI)}

ESCRIBIR


\section{Mission}
Le sujet de la mission d’alternance était d'approfondir la culture \emph{DevOps} en mettant en relation avec Ansible, Docker, Jenkins et le cloud avec Amazon AWS sur un rythme Agile \textit{Scrumban} en ayant Python comme langage de base.

Comme nous l’avons déjà vu dans la Section \ref{bcd}, le complément BCD est celui qui aide à implémenter les bonnes pratiques DevOps au client grâce à l'automatisation de la création d'environnements et l'approvisionnement de stack Bonita, et aussi la compilation et le déploiement d'une application Bonita.

Pour continuer, nous allons voir les concepts \emph{DevOps} ciblés par l'applicatif pour donner un cadre théorique, puis on verra plus en détail le travail réalisé cette année d'alternance.

\subsection{Devops}\label{sec:devops}
De nos jours, dans le domaine du numérique, nous livrons une large gamme de services grâce à des expériences numériques, et pour les améliorer, nous devons avoir le retour du client pour pouvoir proposer de nouveaux types d’expériences et de valeurs. Ce faisant, nous allons être plus compétitifs et nous aurons des nouveaux moyen d'interagir avec les clients.

Tout cela, nous pouvons le faire grâce à des bonnes pratiques pour rendre le travail beaucoup plus efficace, pour cela nous allons entrer dans le monde DevOps.

DevOps est avant tout une culture, des pratiques collaboratives et de l'automatisation qui alignent les équipes de développement et d'opération afin qu'elles puissent améliorer leur expérience client, répondre plus rapidement aux besoins et assurer l'innovation \cite{IsaacSacolick2016DrivingCulture}.

Cependant, cela n'est pas seulement une culture, DevOps involucre principalement un changement de base de comment nous gérons nos applications et comment nous les déployions, en investissant dans l'automation et en changeant les application héritées \cite{benjamin_wootton}.

Pour cette raison, quand nous parlons de DevOps, nous avons:

\begin{itemize}
\item L'automatisation de test.
\item Assurer des déploiements plus fréquents et plus fiables.
\item Avoir la clarté de rôles et des responsabilités.
\item Définir l'indicateur de performance \(KPI\) et sa portée.
\item Automatisation et surveillance.
\end{itemize}

Pour commencer, nous devons nous assurer que tous sont alignés sur ce que DevOps signifie et avec un défi et non à l’ensemble des pratiques DevOps.

Nous devons également rester concentrés sur l’assurance qualité (QA) en sachant qu’il s’agit d’une discipline et d’un ensemble de compétences distincts.

Par exemple, dans la Figure \ref{fig:devops_operational_model} nous voyons une modèle opérationnel proposé ou dans la partie supérieur, l'équipe de \textbf{Dev} qui travaille en \textit{Sprint} en développant les \textit{Epics} avec l'objectif de livrer (Release) les nouvelles fonctionnalités. Dans l'autre côté, l'équipe de \textit{Ops} centré en les issues, l'architecture, la sécurité et les transmettre comme défauts (Defects).


\begin{figure}[!ht]
\centering
\includegraphics[width=\textwidth,keepaspectratio]{devops_operational_model.png}
\caption{Modèle Opérationnel \cite{IsaacSacolick2016DrivingCulture}}
\label{fig:devops_operational_model}
\end{figure}

\begin{figure}[!ht]
\centering
\includegraphics[width=\textwidth,keepaspectratio]{jira_epics_bl.png}
\caption{Exemple de Epics JIRA BCD}
\label{fig:example_epic}
\end{figure}



\subsubsection{Acteurs}
Dans la Figure \ref{fig:devops_operational_model} nous voyons trois acteurs principaux dans cette conversation:

\subparagraph{QA}
C'est une équipe composée par des personnes avec distinctes disciplines qui doivent travailler avec \textit{Dev} pour développer et automatiser les test.
\subparagraph{Dev} L'équipe doit livrer les livrables avec des \emph{runbooks} et d'assurer que les amélioration opérationnelles sont priorité.
\begin{itemize}
  \item Le 30\% d'un sprint doit cibler des défets téchniques.
  \item Cibler le \emph{développement complet}, c'est-à-dire, le développement prêt pour les tests de \textit{QA}.
  \item Collabore activement avec \textit{Ops}.
  \item Assurer que l'application récolte des données qui aident à l'amélioration.
\end{itemize}
\subparagraph{Ops}
L'équipe doit fournir des services dans le cloud pour permettre \textit{Dev} être plus agile et lui permettre d'apprendre à résoudre la plupart des problèmes de production.


\subsection{Le Défi}
Dans la Section \ref{sec:devops}, nous avons listé les activités de transition DevOps qui sont aussi des pratiques que \emph{BCD} vise à donner aux clients de Bonita. Figure \ref{fig:dev-to-prod}

\begin{figure}[!ht]
\centering
\includegraphics[width=\textwidth,keepaspectratio]{dev-to-prod.png}
\caption{Dev à Prod}
\label{fig:dev-to-prod}
\end{figure}

Les technologies utilisées actuellement sont:
\begin{itemize}
  \item Ansible
  \item Python
  \item Docker
  \item Jenkins
  \item Cloud (AWS)
\end{itemize}

Pour cette mission, j'ai dû approfondir mes connaissances sur:
\subparagraph{DevOps} De nouvelles approches qui incluent l'Intégration Continue (CI), le Déploiement Continu (CD).
\subparagraph{Agile} La communication dans un environnement agile avec ses rituels.
\subparagraph{Python} Comme un langage complet avec le paradigme OO avec TDD comme méthodologie.
\subparagraph{Docker} Les bonnes pratiques pour créer une image, et comment nous pouvons les utiliser pour déployer plus rapidement des applications.
\subparagraph{Jenkins} L’architecture et le langage DSL pour  créer des \enquote{jobs} et \enquote{pipelines} pour automatiser les tests, la compilation, l’empaquetage d'un produit final.
\subparagraph{AWS} Tous les nombreux services et la façon de nous en servir.


\subsection{Chronologie}
Tous les livraison de tous les produits suivent le "Software Development life cycle" (SDLC). Chaque six mois est livré une version et une fois par mois est livré une version de maintenance.

\begin{figure}[!ht]
\centering
\includegraphics[scale=0.45]{bcd_my_chrono.png}
\caption{Cycle de vie}
\label{bcd_my_chrono}
\end{figure}

La Figure \ref{bcd_my_chrono} présente la période où j'ai fait ma contribution. Les deux première mois (septembre et octobre), j'ai suivi des formation sur Bonita et j'ai eu le temps pour aborder plus en détail tous les parties du produit \emph{BCD} et les technologies utilisés.


\subsection{Travail réalisé}
Nous avons vu dans dans la Figure \ref{frame_safe} le chemin suivi pour la création et la définition des \textit{Epics}, \textit{Millestones} et le \textit{Backlog}. Dans l'appendice \ref{appen:backlog} nous pouvons regarder un exemple du contenu Backlog.

Pendant le déroulement de mon alternance, j'ai travaillé sur plusieurs items du \textit{Backlog}. Figure \ref{fig:my_issues}. Chaque item était bien découpé en sorte que la tâche puisse se terminer dans le délai d'un \textit{Sprint}.


Pour cela, je devais effectuer des actions différentes comme:

\subparagraph{Python} Développer et maintenir les scripts du Command Line Interface (CLI) de BCD.
\begin{itemize}
  \item J'ai participé à la réorganisation du code pour le rendre plus modulaire.
  \item J'ai participé à l'élaboration des test unitaires pour le module BCD.
  \item J'ai participé à le développement du module de "Licensing".
\end{itemize}

\subparagraph{AWS} Un partie de BCD gère des instances dans le cloud, aussi l'entreprise a passé d'autres services sur le cloud.
\begin{itemize}
  \item J'ai aidé implémenter le service "Organization" de AWS.
  \item J'ai du m'impliquer plus dans les services EC2, Lambda, S3, RDS.
\end{itemize}

\subparagraph{Ansible} Développer et maintenir des "Playbook" Ansible.
\begin{itemize}
  \item J'ai participé à l'adaptation des scripts pour supporter le nouveau dépôt privé d'images docker.
\end{itemize}

\subparagraph{Jenkins} Exécution de jobs pour l'empaquetage de livrables.

\subparagraph{Git} Gestion du dépôt private.
\begin{itemize}
  \item Revue de "Pull Request"
  \item "Merge" dans la branche correspondante.
  \item "Tag" et livre des nouvelles version.
\end{itemize}

\subparagraph{Activités collatérales} Pour mieux organiser le travail en équipe, nous suivons des rituels Agile
\begin{itemize}
  \item Le "Daily Meeting" pour faire le point sur les tâches effectuée et communiquer sur les difficultés rencontrée.
  \item "Rétrospective", pour s’améliore.
  \item "Réunion hebdomadaire",  elle réunit toutes les parties prenantes du projet pour faire le point sur son avancement. Ainsi, tous les participants peuvent se rendre compte si l’état du projet correspond à leurs besoins, attentes ou objectifs.
  \item "Les estimations", où nous faisons des hypothèses à échelle relative, pour estimer une charge de travail par exemple. Cette réunion est importante pour déterminer correctement la complexité et la valeur apporté de la tâche à réaliser pour effectuer une estimation de bonne qualité.
  \item "La sélection", où nous déterminons efficacement la charge de travail à accomplir définis dans le backlog lors de l'itération.
\end{itemize}


\subsection{Travial à faire}
blabla


\subsection{Les Compétences acquises}
Nous avons dans la Section \ref{sec:work_done} les outils et technologies que j'ai eu l'opportunité de toucher du doigt.
Pourtant, je ne peux pas dire que j'ai approfondi seulement le côté technologique et technique, parallèlement, j'ai amélioré mes aptitudes communicatives.


\subsection{Bilan}
Cette période d'alternance chez Bonitasoft m'a apporté une nouvelle expérience professionnelle. Grâce à cette année, j'ai acquis de nouvelles connaissances autant dans le domaine fonctionnel que dans les techniques.

Parmi les difficultés rencontrées, le travail sur un code qui existait déjà avec un langage que je ne connaissais pas en profondeur. L'acquisition des connaissances sur les nouvelles techniques et  technologies ont été aussi une entrave mais à la fois, un défi intéressant et enrichissant qui m'a beaucoup plu.

Cette expérience m'a permis d'affuter mes compétences grâce à l'intégration dans une équipe humaine et dynamique qui a été un véritable catalyseur et une source inépuisable de connaissances et d'expériences.


\subsection{Réflexion finale}
Cette formation en alternance m'a permis d'obtenir beaucoup de compétences et de savoir-faire, que ce soit sur le plan informatique avec la multitude de nouvelles technologies abordées, ceci s'ajoutant aux connaissances déjà acquises que j'ai pu approfondir. Mais également sur le plan personnel, toute la confiance qui, dès mon arrivée, m'a été donnée par l'équipe ; ma liberté d'expression et de prise de décisions ainsi que le partage d'expériences avec les collègues ont été très enrichissants et formateurs.



\section{Réflexion final}
Cette formation en alternance m'a permi d'obtenir beaucoup de compétences et des
savoir-faire que ce soit sur le plan informatique avec la multitude de nouvelles
technologies abordées s'ajoutant aux connaissances déjà acquises que j'ai pu
approfondir. Mais également sur le plan personnel, toute la confiance qui m'a
été donnée dès mon arrivée par l'équipe, ma liberté d'expression et de prise de
décisions ainsi que le partage d'expériences avec les collegues on été très
enrichissants et formateurs.

\subsubsection{Bilan professionnel}

\subsubsection{Bilan personnel}


\subsection{Le Défi}
Dans la Section \ref{sec:devops}, nous avons listé les activités de transition DevOps qui sont aussi des pratiques que \emph{BCD} vise à donner aux clients de Bonita. Figure \ref{fig:dev-to-prod}

\begin{figure}[!ht]
\centering
\includegraphics[width=\textwidth,keepaspectratio]{dev-to-prod.png}
\caption{Dev à Prod}
\label{fig:dev-to-prod}
\end{figure}

Les technologies utilisées actuellement sont:
\begin{itemize}
  \item Ansible
  \item Python
  \item Docker
  \item Jenkins
  \item Cloud (AWS)
\end{itemize}

Pour cette mission, j'ai dû approfondir mes connaissances sur:
\subparagraph{DevOps} De nouvelles approches qui incluent l'Intégration Continue (CI), le Déploiement Continu (CD).
\subparagraph{Agile} La communication dans un environnement agile avec ses rituels.
\subparagraph{Python} Comme un langage complet avec le paradigme OO avec TDD comme méthodologie.
\subparagraph{Docker} Les bonnes pratiques pour créer une image, et comment nous pouvons les utiliser pour déployer plus rapidement des applications.
\subparagraph{Jenkins} L’architecture et le langage DSL pour  créer des \enquote{jobs} et \enquote{pipelines} pour automatiser les tests, la compilation, l’empaquetage d'un produit final.
\subparagraph{AWS} Tous les nombreux services et la façon de nous en servir.


\subsection{Skills and Teachings}
blabla

\subsection{Skills and Teachings}
blabla

\subsection{Travial à faire}
blabla



\newpage
\addcontentsline{toc}{section}{Références}
\newpage

\begin{appendices}
\section{Back log}
Jira Backlog

\newpage
\section{Contribution}
GITHUB
\end{appendices}
\textbf{}
\bibliographystyle{plain}
\bibliography{biblio}
\end{document}
