\subsection{Tous ensemble}

Toutes les équipes dans l'entreprise travaillent à un rythme Agile, et chaque équipe choisit l'implémentation qui lui convient le mieux.

D'une façon globale, Bonitasoft travaille avec le framework représenté dans la Figure \ref{frame_safe}

\begin{figure}[!ht]
\centering
\includegraphics[width=\textwidth,keepaspectratio]{safe_framework.png}
\caption{Cadre de Travail \cite{safe}}
\label{frame_safe}
\end{figure}

Nous voyons dans la partie supérieure que la vision, les fonctionnalités et l'architecture de haut niveau sont définis dans le Comité Produit et le CEO.
Au milieu, le portfolio est traité et organisé par le Product Manager et à la fin, il est développé par l'équipe de développeurs avec un Product Owner et le Scrum Master qui dirige et le QA qui assure la qualité du code.

Dans la Figure \ref{fig:example_epic}, nous voyons un exemple des "Epics" avec les "Issues" liés.
\begin{figure}[!ht]
\centering
\includegraphics[width=\textwidth,keepaspectratio]{jira_epics_bl.png}
\caption{Exemple de Epics JIRA BCD}
\label{fig:example_epic}
\end{figure}
