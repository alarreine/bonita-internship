\subsection{Bonita Continuous Delivery (BCD)} \label{bcd}
Bonita Continuos Delivery ou \textit{BCD} est une solution qui permet l'intégration continue d'une application Bonita en promouvant aussi des bonnes pratiques \emph{DevOps}

Au début, BCD a démarré à partir d'un besoin interne de tester plus facilement des développements. La mise en place d'un environnement avec l'installation de \textit{Bonita Engine} était un processus lent et répétitif, aussi, suite à divers retours des clients, la direction a décidé de voir comme un potentiel produit.

Actuellement, \textit{BCD} a deux parties. Voir Figure \ref{fig:bcd_cap}:
\begin{itemize}
  \item Livingapp, qui permet la compilation d'un projet avec la gestion de bibliothèques essentielles pour une application \textit{Bonita Subscription}, sans avoir besoin de Bonita Studio, et aussi de déployer l'application compilé dans un stack Bonita configuré.
  \item Stack, qui permet d'un part la création des environnements virtuels dans le cloud (AWS actuellement), et l’installation d'un Stack Bonita dans les serveurs configurés.
\end{itemize}

\begin{figure}[!ht]
\centering
\includegraphics[width=\textwidth,keepaspectratio]{bcd_capabilities.png}
\caption{Fonctionnalités BCD}
\label{fig:bcd_cap}
\end{figure}
