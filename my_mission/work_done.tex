\subsection{Travail réalisé}
Nous avons vu dans dans la Figure \ref{frame_safe} le chemin suivi pour la création et la définition des \textit{Epics}, \textit{Millestones} et le \textit{Backlog}. Dans l'appendice \ref{appen:backlog} nous pouvons regarder un exemple du contenu Backlog.

Pendant le déroulement de mon alternance, j'ai travaillé sur plusieurs items du \textit{Backlog}. Figure \ref{fig:my_issues}. Chaque item était bien découpé en sorte que la tâche puisse se terminer dans le délai d'un \textit{Sprint}.


Pour cela, je devais effectuer des actions différentes comme:

\subparagraph{Python} Développer et maintenir les scripts du Command Line Interface (CLI) de BCD.
\begin{itemize}
  \item J'ai participé à la réorganisation du code pour le rendre plus modulaire.
  \item J'ai participé à l'élaboration des test unitaires pour le module BCD.
  \item J'ai participé à le développement du module de "Licensing".
\end{itemize}

\subparagraph{AWS} Un partie de BCD gère des instances dans le cloud, aussi l'entreprise a passé d'autres services sur le cloud.
\begin{itemize}
  \item J'ai aidé implémenter le service "Organization" de AWS.
  \item J'ai du m'impliquer plus dans les services EC2, Lambda, S3, RDS.
\end{itemize}

\subparagraph{Ansible} Développer et maintenir des "Playbook" Ansible.
\begin{itemize}
  \item J'ai participé à l'adaptation des scripts pour supporter le nouveau dépôt privé d'images docker.
\end{itemize}

\subparagraph{Jenkins} Exécution de jobs pour l'empaquetage de livrables.

\subparagraph{Git} Gestion du dépôt private.
\begin{itemize}
  \item Revue de "Pull Request"
  \item "Merge" dans la branche correspondante.
  \item "Tag" et livre des nouvelles version.
\end{itemize}

\subparagraph{Activités collatérales} Pour mieux organiser le travail en équipe, nous suivons des rituels Agile
\begin{itemize}
  \item Le "Daily Meeting" pour faire le point sur les tâches effectuée et communiquer sur les difficultés rencontrée.
  \item "Rétrospective", pour s’améliore.
  \item "Réunion hebdomadaire",  elle réunit toutes les parties prenantes du projet pour faire le point sur son avancement. Ainsi, tous les participants peuvent se rendre compte si l’état du projet correspond à leurs besoins, attentes ou objectifs.
  \item "Les estimations", où nous faisons des hypothèses à échelle relative, pour estimer une charge de travail par exemple. Cette réunion est importante pour déterminer correctement la complexité et la valeur apporté de la tâche à réaliser pour effectuer une estimation de bonne qualité.
  \item "La sélection", où nous déterminons efficacement la charge de travail à accomplir définis dans le backlog lors de l'itération.
\end{itemize}
