\subsection{Devops}\label{sec:devops}
De nos jours, dans le domaine du numérique, nous livrons une large gamme de services grâce à des expériences numériques, et pour les améliorer, nous devons obtenir le retour du client pour pouvoir proposer de nouveaux types d’expériences et de valeurs. Ce faisant, nous allons être plus compétitifs et nous aurons des nouveaux moyens d'interagir avec les clients.

Nous pouvons faire tout cela grâce à de bonnes pratiques pour rendre le travail beaucoup plus efficace, pour cela nous allons entrer dans le monde DevOps.

DevOps est avant tout une culture, qui, avec l'aide des pratiques collaboratives et de l'automatisation, alignent les équipes de développement et d'opération afin qu'elles puissent améliorer leur expérience client, répondre plus rapidement aux besoins et assurer l'innovation \cite{IsaacSacolick2016DrivingCulture}.

Cependant, cela n'est pas seulement une culture, DevOps implique principalement un changement de base de la façon dont nous gérons nos applications et les déployions, aussi, en investissant dans l'automatisation et en changeant les application héritées \cite{benjamin_wootton}.

Pour cette raison, quand nous parlons de DevOps, nous devons parler de:

\begin{itemize}
\item Automatiser les tests.
\item Assurer des déploiements plus fréquents et plus fiables.
\item Avoir la clarté de rôles et des responsabilités.
\item Définir l'indicateur de performance \(KPI\) et sa portée.
\item Surveiller les indicateurs.
\end{itemize}

Pour commencer, nous devons nous assurer que toutes les parties sont alignées sur ce que DevOps signifie comme un défi et non à tout l’ensemble des pratiques DevOps.

Nous devons également rester concentrés sur l’assurance qualité (QA) en sachant qu’il s’agit d’une discipline et d’un ensemble de compétences distincts.

Par exemple, dans la Figure \ref{fig:devops_operational_model} nous voyons un modèle opérationnel proposé. Dans la partie supérieure, l'équipe de \textbf{Dev} qui travaille en \textit{Sprint} en développant les \textit{Epics} avec l'objectif de livrer (Release) les nouvelles fonctionnalités. De l'autre côté, l'équipe de \textbf{Ops} centrée sur les tâches (Issues), l'architecture, la sécurité, et renseigner les défauts (Defects) à l'équipe de \textbf{Dev}.


\begin{figure}[!ht]
\centering
\includegraphics[width=\textwidth,keepaspectratio]{devops_operational_model.png}
\caption{Modèle Opérationnel \cite{IsaacSacolick2016DrivingCulture}}
\label{fig:devops_operational_model}
\end{figure}


\subsubsection{Acteurs}
Dans la Figure \ref{fig:devops_operational_model} nous voyons trois acteurs principaux dans cette conversation:

\subparagraph{QA}
C'est une équipe composée de personnes avec des disciplines distinctes qui doivent travailler avec le \textit{Dev} pour développer et automatiser les tests.
\subparagraph{Dev} L'équipe doit fournir les livrables avec des \emph{runbooks} et s’assurer que les améliorations opérationnelles sont la priorité.
\begin{itemize}
  \item Le 30\% d'un sprint doit cibler des défauts techniques.
  \item Cibler le \emph{développement complet}, c'est-à-dire, le développement prêt pour les tests de \textit{QA}.
  \item Collabore activement avec \textit{Ops}.
  \item Assurer que l'application récolte des données qui aident à l'amélioration.
\end{itemize}
\subparagraph{Ops}
L'équipe doit fournir des services dans le cloud pour permettre \textit{Dev} d’être plus agile et lui permettre d'apprendre à résoudre la plupart des problèmes de production.
